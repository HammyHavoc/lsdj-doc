\chapter{SRAM Memory Map}

This chapter is intended for programmers. It contains information about how song data is stored in SRAM. If you use an emulator, SRAM contents are usually stored in a .sav file.

\section{Bank Layouts}

\subsection{Bank 0}

\begin{description}
    \item[A000-AFEF] phrase notes
    \item[AFF0-B02F] bookmarks
    \item[B030-B08F] empty
    \item[B090-B28F] grooves
    \item[B290-B68F] song chains
    \item[B690-B88F] table envelopes
    \item[B890-BDCF] speech instrument words (\$20*42)
    \item[BDD0-BE77] speech instrument wordnames
    \item[BE78-BE79] mem initialized flag (set to "rb" on init)
    \item[BE7A-BFB9] instrument names
\end{description}

\subsection{Bank 1}

\begin{description}
    \item[A000-A01F] empty
    \item[A020-A03F] table allocation table
    \item[A040-A07F] instr alloc table
    \item[A080-A87F] chain phrases
    \item[A880-B07F] chain transposes
    \item[B080-B47F] instrument parameters \label{sram-instr-param}
    \item[B480-B67F] table transposes
    \item[B680-B87F] table commands, left column
    \item[B880-BA7F] table command values, left column
    \item[BA80-BC7F] table commands, right column
    \item[BC80-BE7F] table command values, right column
    \item[BE80-BE81] mem initialized flag (set to "rb" on init)
    \item[BE82-BEA1] phrase allocation table
    \item[BEA2-BEB1] chain allocation table
    \item[BEB2-BFB1] softsynth params
    \item[BFB2-BFB2] clock, hours
    \item[BFB3-BFB3] clock, minutes
    \item[BFB4-BFB4] tempo
    \item[BFB5-BFB5] tune setting
    \item[BFB6-BFB6] total clock, days
    \item[BFB7-BFB7] total clock, hours
    \item[BFB8-BFB8] total clock, minutes
    \item[BFB9-BFB9] total clock
    \item[BFBA-BFBA] key delay
    \item[BFBB-BFBB] key repeat
    \item[BFBC-BFBC] font (for cgb)
    \item[BFBD-BFBD] sync mode (off/lsdj/midi/...)
    \item[BFBE-BFBE] colorset
    \item[BFBF-BFBF] sync parameter (e.g. analog in ticks/step)
    \item[BFC0-BFC0] clone (0=deep, 1=slim)
    \item[BFC1-BFC1] file changed?
    \item[BFC2-BFC2] power save
    \item[BFC3-BFC3] prelisten
    \item[BFC4-BFC5] synths locked?
    \item[BFC6-BFC9] last used instrument/channel
\end{description}

\subsection{Bank 2}

\begin{description}
    \item[A000-AFEF] phrase commands
    \item[AFF0-BFDF] phrase command values
\end{description}

\subsection{Bank 3}

\begin{description}
    \item[A000-AFFF] waves
    \item[B000-BFEF] phrase instruments
    \item[BFF0-BFF1] mem initialized flag (set to "rb" on init)
    \item[BFF2-BFFE] empty
    \item[BFFF-BFFF] version byte
\end{description}

\section{Instrument Layouts}

Tables \ref{tab:sram-pulse} to \ref{tab:sram-noise} show the SRAM layout of the different instrument types. Each instrument takes up 16 bytes of the instrument parameter area in \nameref{sram-instr-param}.

\begin{table}
	\begin{center}
	\caption{Pulse Instrument SRAM Layout.}
	\label{tab:sram-pulse}
	\begin{tabular}{r|r|l|l}
		\toprule
    Byte & Bits & Content & Value \\
    \midrule
    0  & 7-0 & 0 & \\
    1  & 7-0 & \textsc{Adsr 1} 		& \\
    2  & 7-0 & \textsc{Pu2 Tsp.} 	& \\
    3  & 6   & \textsc{Length=Unlim} 	& 0=yes, 1=no \\
       & 5-0 & \textsc{Length} 		& \\
    4  & 6-0 & \textsc{Sweep} 		& \\
    5  & 7   & \textsc{Pitch=Step} 	& 0=no, 1=yes \\
       & 6   & \textsc{Pitch=Drum} 	& 0=no, 1=yes \\
       & 5   & \textsc{Transpose} 	& 0=on, 1=off \\
       & 4   & \textsc{Pitch} speed	& 0=fast, 1=tick \\
       & 3   & \textsc{Table} speed	& 0=tick, 1=step \\
       & 2-1 & Vibrato shape		& 0=sin, 1=saw, 2=sqr \\
       & 0   & Vibrato direction	& 0=down, 1=up \\
    6  & 5   & Table off/on		& 0=off, 1=on \\
       & 4-0 & Table			& \\
    7  & 7-6 & \textsc{Wave}		& \\
       & 5-2 & \textsc{Finetune}	& \\
       & 1-0 & \textsc{Output}		& \\
    8  & 3-0 & \textsc{Cmd/Rate}	& \\
    9  & 7-0 & \textsc{Adsr 2}		& \\
    10 & 7-0 & \textsc{Adsr 3}		& \\
    \bottomrule
	\end{tabular}
	\end{center}
\end{table}

\begin{table}
	\begin{center}
		\caption{Wave Instrument SRAM Layout.}
		\begin{tabular}{r|r|l|l}
			\toprule
    Byte & Bits & Content & Value \\
    \midrule
    0  & 7-0 & 1 & \\
    1  & 6-5 & \textsc{Volume} 		& \\
    2  & 3-0 & \textsc{Loop Pos} 	& \\
    3  & 7-0 & \textsc{Wave} 		& \\
       & 7-4 & \textsc{Synth} 		& \\
    4  & -   & - & \\
    5  & 7   & \textsc{Pitch=Step} 	& 0=no, 1=yes \\
       & 6   & \textsc{Pitch=Drum} 	& 0=no, 1=yes \\
       & 5   & \textsc{Transpose} 	& 0=on, 1=off \\
       & 4   & \textsc{Pitch} speed	& 0=fast, 1=tick \\
       & 3   & \textsc{Table} speed	& 0=tick, 1=step \\
       & 2-1 & Vibrato shape		& 0=sin, 1=saw, 2=sqr \\
       & 0   & Vibrato direction	& 0=down, 1=up \\
    6  & 5   & Table off/on		& 0=off, 1=on \\
       & 4-0 & Table			& \\
    7  & 1-0 & \textsc{Output}		& \\
    8  & 3-0 & \textsc{Cmd/Rate}	& \\
    9  & 1-0 & \textsc{Playtype}	& 0=off, 1=once, 2=loop, 3=pingpong \\
    10 & 3-0 & \textsc{Length}		& \\
    11 & 7-0 & \textsc{Speed}		& \\
    12 & 7-3 & \textsc{Finetune}	& \\
    \bottomrule
		\end{tabular}
	\end{center}
\end{table}

\begin{table}
	\begin{center}
		\caption{Kit Instrument SRAM Layout.}
		\begin{tabular}{r|r|l|l}
			\toprule
   Byte & Bits	& Content 		& Value \\
   \midrule
      0	& 7-0 	& 2 			& \\
      1	& 6-5 	& \textsc{Volume} 	& \\
      2	& 7   	& \textsc{Loop 1=Atk} 	& 0=no, 1=yes \\
	& 6   	& \textsc{Speed=0.5x} 	& 0=no, 1=yes \\
	& 5-0 	& \textsc{Kit} 		& \\
      3	& 7-0 	& \textsc{Length 1} 	& 0=\textsc{Auto} \\
      4	& -   	& - 			& \\
      5	& 7   	& \textsc{Pitch=Step} 	& 0=no, 1=yes \\
	& 6   	& \textsc{Loop 1} 	& 0=off, 1=on \\
	& 5   	& \textsc{Loop 2} 	& 0=off, 1=on \\
	& 4   	& \textsc{Pitch} speed	& 0=fast, 1=tick \\
	& 3   	& \textsc{Table} speed	& 0=tick, 1=step \\
	& 2-1 	& Vibrato shape		& 0=sin, 1=saw, 2=sqr \\
	& 0   	& Vibrato direction	& 0=down, 1=up \\
      6	& 5   	& Table off/on		& 0=off, 1=on \\
	& 4-0 	& Table			& \\
      7	& 1-0 	& \textsc{Output}	& \\
      8	& 7-0 	& \textsc{Finetune}	& \\
      9	& 7   	& \textsc{Loop 2=Atk} 	& 0=no, 1=yes \\
	& 5-0 	& \textsc{Kit} 		& \\
     10	& 7-0 	& \textsc{Dist} 	& \$d0-\$d3=\textsc{clip, shape, shap2, wrap} \\
     11	& 7-0 	& \textsc{Length 2} 	& 0=\textsc{Auto} \\
     12	& 7-0 	& \textsc{Offset 1} 	& \\
     13	& 7-0 	& \textsc{Offset 2} 	& \\
   \bottomrule
		\end{tabular}
	\end{center}
\end{table}

\begin{table}
	\begin{center}
		\caption{Noise Instrument SRAM Layout.}
		\label{tab:sram-noise}
		\begin{tabular}{r|r|l|l}
			\toprule
   Byte	& Bits	& Content 		& Value \\
\midrule
      0	& 7-0 	& 3 			& \\
      1 & 7-0 	& \textsc{Adsr 1} 	& \\
      2 & -   	& -  		 	& \\
      3 & 6   	& \textsc{Length=Unlim}	& 0=yes, 1=no \\
        & 5-0 	& \textsc{Length} 	& \\
      4 & 7-0 	& Last entered note 	& \\
      5 & 3     & \textsc{Table} speed	& 0=tick, 1=step \\
      6	& 5   	& Table off/on		& 0=off, 1=on \\
	& 4-0 	& Table			& \\
      7 & 1-0 	& \textsc{Output}	& \\
      8 & 3-0 	& \textsc{Cmd/Rate}	& \\
      9 & 7-0 	& \textsc{Adsr 2}	& \\
     10 & 7-0 	& \textsc{Adsr 3}	& \\
     \bottomrule
		\end{tabular}
	\end{center}
\end{table}
